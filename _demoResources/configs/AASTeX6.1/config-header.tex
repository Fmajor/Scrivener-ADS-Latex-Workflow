%% Reintroduced the \received and \accepted commands from AASTeX v5.2

%% \documentclass{aastex61}
%% \documentclass[twocolumn,linenumbers,trackchanges]{aastex61}

\documentclass[twocolumn, trackchanges]{../_demoResources/configs/AASTex6.1/aastex61}
\renewcommand\equationautorefname{Eq. }
\renewcommand\subsubsectionautorefname{Section}
\renewcommand\subsectionautorefname{Section}
\renewcommand\sectionautorefname{Section}

\received{July 1, 2016}
\revised{September 27, 2016}
\accepted{\today}
%% Command to document which AAS Journal the manuscript was submitted to.
%% Adds "Submitted to " the arguement.
\submitjournal{ApJ}

%% Mark up commands to limit the number of authors on the front page.
%% Note that in AASTeX v6.1 a \collaboration call (see below) counts as
%% an author in this case.
%
%\AuthorCollaborationLimit=3
%
%% Will only show Schwarz, Muench and "the AAS Journals Data Scientist 
%% collaboration" on the front page of this example manuscript.
%%
%% Note that all of the author will be shown in the published article.
%% This feature is meant to be used prior to acceptance to make the
%% front end of a long author article more manageable. Please do not use
%% this functionality for manuscripts with less than 20 authors. Conversely,
%% please do use this when the number of authors exceeds 40.
%%
%% Use \allauthors at the manuscript end to show the full author list.
%% This command should only be used with \AuthorCollaborationLimit is used.

%% The following command can be used to set the latex table counters.  It
%% is needed in this document because it uses a mix of latex tabular and
%% AASTeX deluxetables.  In general it should not be needed.
%\setcounter{table}{1}

%%%%%%%%%%%%%%%%%%%%%%%%%%%%%%%%%%%%%%%%%%%%%%%%%%%%%%%%%%%%%%%%%%%%%%%%%%%%%%%%
%%
%% The following section outlines numerous optional output that
%% can be displayed in the front matter or as running meta-data.
%%
%% If you wish, you may supply running head information, although
%% this information may be modified by the editorial offices.
\shorttitle{\aastex\ sample article}
\shortauthors{Schwarz et al.}
%%
%% You can add a light gray and diagonal water-mark to the first page 
%% with this command:
% \watermark{text}
%% where "text", e.g. DRAFT, is the text to appear.  If the text is 
%% long you can control the water-mark size with:
%  \setwatermarkfontsize{dimension}
%% where dimension is any recognized LaTeX dimension, e.g. pt, in, etc.
%%
%%%%%%%%%%%%%%%%%%%%%%%%%%%%%%%%%%%%%%%%%%%%%%%%%%%%%%%%%%%%%%%%%%%%%%%%%%%%%%%%
%% This is the end of the preamble.  Indicate the beginning of the
%% manuscript itself with \begin{document}.

\begin{document}
	
	\title{An Example Article using \aastex v6.1\footnote{This version fixes many bugs from v6.0 and introduces some new features, primarily in the way the author and affiliations are now marked up.}}
	
	%% LaTeX will automatically break titles if they run longer than
	%% one line. However, you may use \\ to force a line break if
	%% you desire. In v6.1 you can include a footnote in the title.
	
	%% A significant change from earlier AASTEX versions is in the structure for 
	%% calling author and affilations. The change was necessary to implement 
	%% autoindexing of affilations which prior was a manual process that could 
	%% easily be tedious in large author manuscripts.
	%%
	%% The \author command is the same as before except it now takes an optional
	%% arguement which is the 16 digit ORCID. The syntax is:
	%% \author[xxxx-xxxx-xxxx-xxxx]{Author Name}
	%%
	%% This will hyperlink the author name to the author's ORCID page. Note that
	%% during compilation, LaTeX will do some limited checking of the format of
	%% the ID to make sure it is valid.
	%%
	%% Use \affiliation for affiliation information. The old \affil is now aliased
	%% to \affiliation. AASTeX v6.1 will automatically index these in the header.
	%% When a duplicate is found its index will be the same as its previous entry.
	%%
	%% Note that \altaffilmark and \altaffiltext have been removed and thus 
	%% can not be used to document secondary affiliations. If they are used latex
	%% will issue a specific error message and quit. Please use multiple 
	%% \affiliation calls for to document more than one affiliation.
	%%
	%% The new \altaffiliation can be used to indicate some secondary information
	%% such as fellowships. This command produces a non-numeric footnote that is
	%% set away from the numeric \affiliation footnotes.  NOTE that if an
	%% \altaffiliation command is used it must come BEFORE the \affiliation call,
	%% right after the \author command, in order to place the footnotes in
	%% the proper location.
	%%
	%% Use \email to set provide email addresses. Each \email will appear on its
	%% own line so you can put multiple email address in one \email call. A new
	%% \correspondingauthor command is available in V6.1 to identify the
	%% corresponding author of the manuscript. It is the author's responsibility
	%% to make sure this name is also in the author list.
	%%
	%% While authors can be grouped inside the same \author and \affiliation
	%% commands it is better to have a single author for each. This allows for
	%% one to exploit all the new benefits and should make book-keeping easier.
	%%
	%% If done correctly the peer review system will be able to
	%% automatically put the author and affiliation information from the manuscript
	%% and save the corresponding author the trouble of entering it by hand.
	
	\correspondingauthor{August Muench}
	\email{greg.schwarz@aas.org, gus.muench@aas.org}
	
	\author[0000-0002-0786-7307]{Greg J. Schwarz}
	\affil{American Astronomical Society \\
		2000 Florida Ave., NW, Suite 300 \\
		Washington, DC 20009-1231, USA}
	
	\author{August Muench}
	\affiliation{American Astronomical Society \\
		2000 Florida Ave., NW, Suite 300 \\
		Washington, DC 20009-1231, USA}
	\collaboration{(AAS Journals Data Scientists collaboration)}
	
	\author{Butler Burton}
	\affiliation{National Radio Astronomy Observatory}
	\affiliation{AAS Journals Associate Editor-in-Chief}
	\nocollaboration
	
	\author{Amy Hendrickson}
	\altaffiliation{Creator of AASTeX v6.1}
	\affiliation{TeXnology Inc.}
	\collaboration{(LaTeX collaboration)}
	
	\author{Julie Steffen}
	\affiliation{AAS Director of Publishing}
	\affiliation{American Astronomical Society \\
		2000 Florida Ave., NW, Suite 300 \\
		Washington, DC 20009-1231, USA}
	
	\author{Jeff Lewandowski}
	\affiliation{IOP Senior Publisher for the AAS Journals}
	\affiliation{IOP Publishing, Washington, DC 20005}
	
	%% Note that the \and command from previous versions of AASTeX is now
	%% depreciated in this version as it is no longer necessary. AASTeX 
	%% automatically takes care of all commas and "and"s between authors names.
	
	%% AASTeX 6.1 has the new \collaboration and \nocollaboration commands to
	%% provide the collaboration status of a group of authors. These commands 
	%% can be used either before or after the list of corresponding authors. The
	%% argument for \collaboration is the collaboration identifier. Authors are
	%% encouraged to surround collaboration identifiers with ()s. The 
	%% \nocollaboration command takes no argument and exists to indicate that
	%% the nearby authors are not part of surrounding collaborations.